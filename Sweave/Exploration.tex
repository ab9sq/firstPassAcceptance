\documentclass{article}

\usepackage[toc]{multitoc}
\usepackage{datetime}
\renewcommand*{\multicolumntoc}{2}
\setlength{\columnseprule}{0.5pt}


\title{Potential Metric Items From FPA Spreadsheet}
\author{Nick Lauerman}
%\date{23 Jan 2017}


\usepackage{Sweave}
\begin{document}
\input{Exploration-concordance}
\maketitle

\begin{abstract}
Evaulation of various potential matrics, performance indicators for the Software
Quality Assurance collect on the First Pass Acceptance Excel workbook.
\footnote{Date Ran: \today{} at \currenttime} % time not correctly formating needs a space
\end{abstract}

\tableofcontents

\section{Introduction}
The data is processed using R, Version 3.4.3 named Kite-Eating Tree. The only
extension (package or library) utilized is lubridate (version 1.7.1) to provide
key functionality in the processing of dates.

The data is read into R from a comma seperate value (csv) file which is derived
from the spreadsheet without modification. Additional values are computed as needed.

This report is prepared in the R enviroment using a collection of packages know as
Sweave that included knitr which inturn feeds the package into \LaTeXe{} a typeseting
program to produce a PDF file. \LaTeXe{} is utilizing the folowing packages to control
style and formating:
\begin{itemize}
\item datetime and
\item multitoc
\end{itemize}

This will only show point in time data. Or what we did this period.

\section{Data and Calculations}
\subsection{Data}
A seperate data diction will be prepared to for both the source data
and and computed values stored in R.

\subsection{Calculations}
All calculations presented here are not implemented in R as formulas.

\section{Work Yet to be Completed}
\subsection{Formula}
Some calculations would be better implemented as formulas and which matrics those
are needs to be evulated and the formulas developed.
\subsection{Trending}
Eventually the ``System'' will be updated to select the data on a for a calander
month. After processing the data will be stored in a seperate file to provide the
results for that month. This will allow for longr term trending of this data.
\subsection{Graphs}
When ``trending'' is implented graphs of the trend will also be added.

\section{Quanity Metrics}
\subsection{Application supported}
This is a list of all application support this period. This may differ from projects
supported as some application may have multiple projects in a period.
\begin{Schunk}
\begin{Soutput}
 [1] "Groninger"                   "Assay File Database "       
 [3] "PCN/SCN"                     "WWLIMS"                     
 [5] "Pulse"                       "DFCS"                       
 [7] "DPW"                         "QIMS"                       
 [9] "Abbott Transfusion Medicine" "Apollo/PHM"                 
[11] "AFMS"                        "SAS"                        
[13] "Metrics Library"             "DaVinci"                    
\end{Soutput}
\end{Schunk}
\subsection{Number of Projects}
The total number of unique projects worked on this period. This may differ from the
application list becouse some application may have more than one project in a periond.
\begin{Schunk}
\begin{Soutput}
[1] 15
\end{Soutput}
\end{Schunk}

\subsection{Number of reviews}
This is the number of items reviewed, a document may be counted more than once if it is reviewed
more than once.
\begin{Schunk}
\begin{Soutput}
[1] 66
\end{Soutput}
\end{Schunk}

\subsection{Number of items reviewed}
This is the number of unique items reviewed, each item is only counted once regardless of
how many times it is reviewed.

Note: right now this will report low due to certian items having no way to descrimate between
such as multiple executions of the same test script or results from an IIVP run on different
computers within the same project.
\begin{Schunk}
\begin{Soutput}
[1] 57
\end{Soutput}
\end{Schunk}

\subsection{Number of Projects Started}
The point at which a project is started is when version 1 of the \textit{Software
Change Request} is approved.

\begin{Schunk}
\begin{Soutput}
[1] 2
\end{Soutput}
\end{Schunk}

\subsection{Projects Completed}

The point at which a project is considered completed is when version 1 of the
\textit{System Certification Summary} is approved.

\begin{Schunk}
\begin{Soutput}
[1] 0
\end{Soutput}
\end{Schunk}

\section{Rate Metrics}
\subsection{Document Approval Rate}
This is portion of documents that are approved upon review.

It is computed by counting the number of approvals (to include first pass)
and deviding that by the total number of reviews

\begin{Schunk}
\begin{Soutput}
[1] 0.7575758
\end{Soutput}
\end{Schunk}

\subsection{First Pass Acceptance}
this is the portion of documents that are approved on the first review conducted
by SQA.

It is computed by counting the number of first pass approvals and deviding by
the number of reviews

\begin{Schunk}
\begin{Soutput}
[1] 0.5454545
\end{Soutput}
\end{Schunk}

\end{document}
